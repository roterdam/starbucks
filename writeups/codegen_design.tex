\documentclass[12pt]{article}
\usepackage[pdftex]{graphicx}
\usepackage[table]{xcolor}
\usepackage{amsfonts}
\usepackage{amssymb}
\usepackage{amsmath}
\usepackage{latexsym}
\usepackage{enumerate}
\setlength{\parskip}{1pc}
\setlength{\parindent}{0pt}
\setlength{\textheight}{9.5in}
\setlength{\textwidth}{6.5in}
\usepackage[left=1.25in, right=1.25in, top=1.25in, bottom=1.25in, includefoot,
headheight=13.6pt]{geometry}
\pagestyle{empty}

\begin{document}

\title{6.035 - Code Generation \\ Design Documentation}
\author{David Chang \\ Saif Hakim \\ Joshua Ma \\Sajith Wickramasekara}
\date{\today}
\maketitle

\section{Introduction}
Currently our compiler generates an abstract syntax tree from an input program and applies a set of
semantic checks to it. The next step is to build lower level representations of the AST such that
assembly translation is straightforward. However, different optimizations are best applied at
different levels of abstraction. Thus, we will first generate a mid-level IR and apply relevant
optimizations before creating a low-level IR that is easily mapped to assembly. We plan to
accomplish this using visitor patterns to traverse and modify our AST.

\subsection{Mid-Level IR}
The purpose of the mid-level intermediate representation is to remove certain abstractions from the
high-level IR. After semantic checking, a good deal of the structure is unnecessary and can be
stripped away. The result more closely resemble assembly code, with the major exception of being
hardware independent.

The focus is on flattening the code tree, eliminating complex control structures (through the use of
jump and label nodes), and (naively) expanding nested expressions. We implement a visitor that
recursively replaces each node in the tree with a linked-list, where the head and tail of the list
are linked to the previous and next siblings respectively.

\subsection{Control Structures}
If, for and while statements are the supported control statements in Decaf, and each can be expanded
into a linear series of expression evaluation, main block, jump node where applicable. The
statements of an if/else statement, then, are no longer grouped into two subtrees. The statements
are in a linear linked list, with expression evaluation and jump logic inserted between.

\subsection{Integer Expressions}
Integer expressions are flattened through the addition of temporary variable storage. Rather than
represent exprA + exprB as a tree (+ exprA exprB), the root node is replaced with three linked
nodes:
\begin{verbatim}
[tempA = resultOfExprA,
tempB = resultOfExprB,
resultOfPlus = tempA+tempB].
\end{verbatim}

The first linked node might be expanded recursively:

\begin{verbatim}
[tempAA = 1,
tempAB = 2,
tempA = tempAA * tempAB,
tempB = resultOfExprB,
resultOfPlus = tempA+tempB].
\end{verbatim}

Each expression is decomposed into operations of at most three operands (one is the assignment
operand). This also requires the design of a namespacing controller, able to allocate
non-conflicting names and look up names given a node.

\subsection{Boolean Expressions}

Boolean expressions are treated differently, due to the required presence of short-circuiting
optimization. Utilizing the algorithm presented in class, jump nodes and block statements are
assembled into a graph structure. The structure is then linearized through converting the pointers
in jump statements to newly created labels.

\subsection{Methods}

In the mid-level IR, each method declaration remains its own linked list, referenced by a global
scope object. Method calls aren’t broken down yet, but rather consist of a node storing pointers to
the method declaration and the nodes where parameters are assigned.

\section{Low-Level IR}

The purpose of our low level IR is to simplify the process of translating our AST to assembly. The
representation will be constructed with the target architecture in mind. It will incorporate the
idea of register and memory allocation. Every node in our AST will maintain a data structure of
instructions with the ability to recursively determine child instructions.
Methods

All methods will be identified by labels and we will introduce nodes that indicate setup and
teardown, producing appropriate assembly code during the final step of code generation. Since space
for local variables is then allocated on the stack, each variable node will be able to indicate how
much space it requires at this point. The visitor keeps track of the variable locations to later
produce offsets. It also maintains positions for the relevant pointers in each scope.
Register and Memory Allocation

Through using a visitor object to generate the low-level IR, the visitor maintains state, including
register allocation. When queried, it provides storage and access requests with an available
register, using the stack if registers are not available. All callee/caller-register logic is
handled by the visitor.
Visitor Design

Throughout this document, a generic “Visitor” object for each IR has been referenced. The actual
visitor objects will be composed of submodules responsible for their individual tasks, coordinated
by one main object.

A final visitor runs through the low-level IR to generate assembly code using the Intel x86 syntax
and NASM.j


\end{document}
